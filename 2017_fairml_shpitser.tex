\documentclass[12pt,t]{beamer}
\usepackage{graphicx}
%\beamerdefaultoverlayspecification{<+->}
%\setbeamercovered{transparent}
\setbeameroption{hide notes}
\setbeamertemplate{note page}[plain]
\usepackage{listings}
\usepackage{datetime}
\usepackage{url}

%make math easier
\usepackage{bm}
\usepackage{amsmath}
\newcommand{\E}{\mathbb{E}}
\newcommand{\D}{\mathcal{D}}
\newcommand{\F}{\mathcal{F}}
\newcommand{\X}{\mathcal{X}}

%Bibliography
\usepackage{natbib}
\bibpunct{(}{)}{,}{a}{}{;}
\usepackage{bibentry}
%\nobibliography*

\input{header.tex}

%%%%%%%%%%%%%%%%%%%%%%%%%%%%%%%%%%%%%%%%%%%%%%%%%%%%%%%%%%%%%%%%%%%%%%
% end of header
%%%%%%%%%%%%%%%%%%%%%%%%%%%%%%%%%%%%%%%%%%%%%%%%%%%%%%%%%%%%%%%%%%%%%%

% title info
\title{Fairness Through Mediation}
\subtitle{\scriptsize for the seminar: \textit{Fairness in Machine Learning} \\
                      organized by M.~Hardt, Fall 2017, UC Berkeley}
\author{\href{http://nimahejazi.org}{Nima Hejazi}
       \\[-10pt]
       }
\institute{Division of Biostatistics \\
           University of California, Berkeley \\
           \href{https://www.stat.berkeley.edu/~nhejazi}
             {\tt \scriptsize \color{foreground}
               stat.berkeley.edu/\textasciitilde{}nhejazi
             }
           \\[4pt]
           \includegraphics[height=20mm]{Figs/seal-berkeley.png}
           \\[-12pt]
          }
\date{
  \href{http://nimahejazi.org}
      {\tt \scriptsize \color{foreground} nimahejazi.org}
  \\[-4pt]
  \href{https://twitter.com/nshejazi}
      {\tt \scriptsize \color{foreground} twitter/@nshejazi}
  \\[-4pt]
  \href{https://github.com/nhejazi}
      {\tt \scriptsize \color{foreground} github/nhejazi}
}

%%%%%%%%%%%%%%%%%%%%%%%%%%%%%%%%%%%%%%%%%%%%%%%%%%%%%%%%%%%%%%%%%%%%%%%%%%%%%%%%

\begin{document}

% title slide
{
\setbeamertemplate{footline}{} % no page number here
\frame{
  \titlepage

  \vspace{-1em}

  \centerline{\href{https://goo.gl/8RWEy5}{\tt \scriptsize
                                           \underline{slides}: goo.gl/8RWEy5}}
  \vspace{-1.5em}
  \vfill \hfill \includegraphics[height=6mm]{Figs/cc-zero.png} \vspace*{-0.5cm}

  \note{This slide deck is for a reading group presentation on the manuscript
    ``Fair Inference on Outcomes'' (Rabi \& Shpitser, 2017), for the seminar on
    ``Fairness in Machine Learning'', organized in Fall 2017 by Moritz Hardt, at
    the University of California, Berkeley.

    Source: {\tt https://github.com/nhejazi/talk\_fair-outcomes} \\
    Slides: {\tt https://goo.gl/i3CxL9} \\
    With notes: {\tt https://goo.gl/8RWEy5}
}
}
}

%%%%%%%%%%%%%%%%%%%%%%%%%%%%%%%%%%%%%%%%%%%%%%%%%%%%%%%%%%%%%%%%%%%%%%%%%%%%%%%%

\begin{frame}[c]{Preview: Summary}
\only<1>{\addtocounter{framenumber}{-1}}

\begin{center}
\begin{itemize}
  \itemsep8pt
  \item Mediation analysis provides a framework under which intuitive
    definitions of fairness may be expressed.
  \item ``Fair inference'' is analogous to causal inference, except in that the
    counterfactuals explored refer to a ``fair'' world (n.~b., intentionally
    vague).
  \item Fairness may be characterized as the absence (or dampening) of a
    \textbf{path-specific effect (PSE)}.
  \item Restriction of a PSE is easily expressed as a likelihood maximization
    problem that features contraining the magnitude of the undesirable PSE.
  \item This approach to fairness allows us to avoid throwing away information
    (e.g., ``I don't see color\dots'') while leaving specific definitions of
    fairness to the analyst.
\end{itemize}
\end{center}

\note{We'll go over this summary again at the end of the talk. Hopefully, it
  will make more sense then.
}

\end{frame}

%%%%%%%%%%%%%%%%%%%%%%%%%%%%%%%%%%%%%%%%%%%%%%%%%%%%%%%%%%%%%%%%%%%%%%%%%%%%%%%%

\begin{frame}[c]{Preliminaries: Notation}

\begin{center}
\begin{itemize}
  \itemsep10pt
  \item Data $\D = (Y, \bm{X})$; outcome $Y$, feature vector $\bm{X}$.
  \item Sensitive features: $S \in \bm{X}$, where inference on $Y$ using $S$
    \textit{might} result in discrimination.
  \item Treatment variable: $A \in \bm{X}$.
  \item Mediator variables: $M \in \bm{X}$ or $M \subseteq \bm{X}$.
  \item Potential outcome: $Y(a)$, realization of $Y$ under $A = a$.
\end{itemize}
\end{center}

\note{...}

\end{frame}

%%%%%%%%%%%%%%%%%%%%%%%%%%%%%%%%%%%%%%%%%%%%%%%%%%%%%%%%%%%%%%%%%%%%%%%%%%%%%%%%

\begin{frame}[c]{Preliminaries: Mediation Analysis}

\begin{center}
\begin{itemize}
  \itemsep10pt
  \item \textbf{Goal:} understand the mechanism by which $A$ influences $Y$.
  \item Decompose the \textbf{ACE} into \textit{direct} and \textit{indirect}
    effects mediated by a variable $M$.
  \item Partition feature space $\bf{X}$ into $A$ (treatment), $M$ (mediator),
    and $\bf{C} = \bf{X}$ \textbackslash \hspace{0.1em} $\{A, M\}$ (baseline
    factors).
  \item Counterfactual contrasts are expressed via \textit{nested} potential
    outcomes (i.e., $Y(a, M(a'))$).
\end{itemize}
\end{center}

\note{
  \begin{itemize}
      \item Nested potential outcomes read as ``the outcome $Y$ if $A$ were set
        to $a$ while $M$ were set to whatever value it would have attained had
        $A$ been set to $a'$.
  \end{itemize}
}

\end{frame}

%%%%%%%%%%%%%%%%%%%%%%%%%%%%%%%%%%%%%%%%%%%%%%%%%%%%%%%%%%%%%%%%%%%%%%%%%%%%%%%%

\begin{frame}[c]{The Average Causal Effect (ACE)}

\begin{center}
\begin{itemize}
  \itemsep10pt
  \item $\text{ACE} = \E[Y(a)] - \E[Y(a')]$
  \item Not computed via $\E[Y \mid A],$ as associations between $A$ and $Y$ may
    be ``partly causal'' or spurious.
  \item Decomposition: $\text{ACE} = \text{NDE} + \text{NIE},$ where
    \textbf{NDE} is the \textit{Natural Direct Effect} and \textbf{NIE} is the
    \textit{Natural Indirect Effect}.
  \item $\text{ACE} = \E[Y(a)] - \E[Y(a')] = \E[Y(a)] - \E[Y(a, M(a')] +
    \E[(Y(a, M(a')] - \E[Y(a')]$
\end{itemize}
\end{center}

\note{
  \begin{itemize}
      \item Decomposition of the ACE gives us a way to express undesirable PSEs
        using mediators
      \item The decomposition is just by way of a telescoping sum (i.e., the old
        ``add and subtract the same thing'' trick).
    \end{itemize}
}

\end{frame}

%%%%%%%%%%%%%%%%%%%%%%%%%%%%%%%%%%%%%%%%%%%%%%%%%%%%%%%%%%%%%%%%%%%%%%%%%%%%%%%%

\begin{frame}[c]{The Natural \textit{Direct} Effect (NDE)}

\begin{center}
\begin{itemize}
  \item ...
  \item ...
\end{itemize}
\end{center}

\note{...}

\end{frame}

%%%%%%%%%%%%%%%%%%%%%%%%%%%%%%%%%%%%%%%%%%%%%%%%%%%%%%%%%%%%%%%%%%%%%%%%%%%%%%%%

\begin{frame}[c]{The Natural \textit{Indirect} Effect (NIE)}

\begin{center}
\begin{itemize}
  \item ...
  \item ...
\end{itemize}
\end{center}

\note{...}

\end{frame}

%%%%%%%%%%%%%%%%%%%%%%%%%%%%%%%%%%%%%%%%%%%%%%%%%%%%%%%%%%%%%%%%%%%%%%%%%%%%%%%%

\begin{frame}[c]{Introducing the Method or Approach, Part I}

\begin{center}
\begin{itemize}
  \item ...
  \item ...
\end{itemize}
\end{center}

\note{...}

\end{frame}

%%%%%%%%%%%%%%%%%%%%%%%%%%%%%%%%%%%%%%%%%%%%%%%%%%%%%%%%%%%%%%%%%%%%%%%%%%%%%%%%

\begin{frame}[c]{Introducing the Method or Approach, Part II}

\begin{center}
\begin{itemize}
  \item ...
  \item ...
\end{itemize}
\end{center}

\note{...}

\end{frame}

%%%%%%%%%%%%%%%%%%%%%%%%%%%%%%%%%%%%%%%%%%%%%%%%%%%%%%%%%%%%%%%%%%%%%%%%%%%%%%%%

\begin{frame}[c]{Introducing the Method or Approach, Part III}

\begin{center}
\begin{itemize}
  \item ...
  \item ...
\end{itemize}
\end{center}

\note{...}

\end{frame}

%%%%%%%%%%%%%%%%%%%%%%%%%%%%%%%%%%%%%%%%%%%%%%%%%%%%%%%%%%%%%%%%%%%%%%%%%%%%%%%%

\begin{frame}[c]{Discussion of Results, Part I}

\begin{center}
\begin{itemize}
  \item ...
  \item ...
\end{itemize}
\end{center}

\note{...}

\end{frame}

%%%%%%%%%%%%%%%%%%%%%%%%%%%%%%%%%%%%%%%%%%%%%%%%%%%%%%%%%%%%%%%%%%%%%%%%%%%%%%%%

\begin{frame}[c]{Discussion of Results, Part II}

\begin{center}
\begin{itemize}
  \item ...
  \item ...
\end{itemize}
\end{center}

\note{...}

\end{frame}

%%%%%%%%%%%%%%%%%%%%%%%%%%%%%%%%%%%%%%%%%%%%%%%%%%%%%%%%%%%%%%%%%%%%%%%%%%%%%%%%

\begin{frame}[c]{Review: Summary}

\begin{center}
\begin{itemize}
  \itemsep8pt
  \item Mediation analysis provides a framework under which intuitive
    definitions of fairness may be expressed.
  \item ``Fair inference'' is analogous to causal inference, except in that the
    counterfactuals explored refer to a ``fair'' world (n.~b., intentionally
    vague).
  \item Fairness may be characterized as the absence (or dampening) of a
    \textbf{path-specific effect (PSE)}.
  \item Restriction of a PSE is easily expressed as a likelihood maximization
    problem that features contraining the magnitude of the undesirable PSE.
  \item This approach to fairness allows us to avoid throwing away information
    (e.g., ``I don't see color\dots'') while leaving specific definitions of
    fairness to the analyst.
\end{itemize}
\end{center}

\note{It's always good to include a summary. You've seen this all before.}

\end{frame}

%%%%%%%%%%%%%%%%%%%%%%%%%%%%%%%%%%%%%%%%%%%%%%%%%%%%%%%%%%%%%%%%%%%%%%%%%%%%%%%%

\begin{frame}[c,allowframebreaks]{References}

\bibliographystyle{apalike}
\nocite{*}
\bibliography{references}

\note{Here's some work we've talked about. Go check these out if interested.}

\end{frame}

%%%%%%%%%%%%%%%%%%%%%%%%%%%%%%%%%%%%%%%%%%%%%%%%%%%%%%%%%%%%%%%%%%%%%%%%%%%%%%%%

\begin{frame}[c]{Thank you.}

\Large
Slides: \href{https://goo.gl/i3CxL9}{goo.gl/i3CxL9} \quad
\includegraphics[height=5mm]{Figs/cc-zero.png}

\vspace{5mm}
Notes: \href{https://goo.gl/8RWEy5}{goo.gl/8RWEy5}

\vspace{5mm}
\href{https://www.stat.berkeley.edu/~nhejazi}{\tt stat.berkeley.edu/\textasciitilde{}nhejazi}

\vspace{5mm}
\href{http://nimahejazi.org}{\tt nimahejazi.org}

\vspace{5mm}
\href{https://twitter.com/nshejazi}{\tt twitter/@nshejazi}

\vspace{5mm}
\href{https://github.com/nhejazi}{\tt github/nhejazi}

\note{Here's where you can find me, as well as the slides for this talk.}

\end{frame}

%%%%%%%%%%%%%%%%%%%%%%%%%%%%%%%%%%%%%%%%%%%%%%%%%%%%%%%%%%%%%%%%%%%%%%%%%%%%%%%%

\end{document}

