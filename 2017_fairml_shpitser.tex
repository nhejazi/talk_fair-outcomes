\documentclass[12pt,t]{beamer}
\usepackage{graphicx}
\setbeameroption{hide notes}
\setbeamertemplate{note page}[plain]
\usepackage{listings}
\usepackage{datetime}
\usepackage{url}

%Bibliography
\usepackage{natbib}
\bibpunct{(}{)}{,}{a}{}{;}
\usepackage{bibentry}
\nobibliography*

\input{header.tex}

%%%%%%%%%%%%%%%%%%%%%%%%%%%%%%%%%%%%%%%%%%%%%%%%%%%%%%%%%%%%%%%%%%%%%%
% end of header
%%%%%%%%%%%%%%%%%%%%%%%%%%%%%%%%%%%%%%%%%%%%%%%%%%%%%%%%%%%%%%%%%%%%%%

% title info
\title{Fairness Through Mediation}
\subtitle{\scriptsize for the seminar: \textit{Fairness in Machine Learning} \\
                      organized by M.~Hardt, Fall 2017, UC Berkeley}
\author{\href{http://nimahejazi.org}{Nima Hejazi}
       \\[-10pt]
       }
\institute{Division of Biostatistics \\
           University of California, Berkeley \\
           \href{https://www.stat.berkeley.edu/~nhejazi}
             {\tt \scriptsize \color{foreground}
               stat.berkeley.edu/\textasciitilde{}nhejazi
             }
           \\[4pt]
           \includegraphics[height=20mm]{Figs/seal-berkeley.png}
           \\[-12pt]
          }
\date{
  \href{http://nimahejazi.org}
      {\tt \scriptsize \color{foreground} nimahejazi.org}
  \\[-4pt]
  \href{https://twitter.com/nshejazi}
      {\tt \scriptsize \color{foreground} twitter/@nshejazi}
  \\[-4pt]
  \href{https://github.com/nhejazi}
      {\tt \scriptsize \color{foreground} github/nhejazi}
}


\begin{document}

% title slide
{
\setbeamertemplate{footline}{} % no page number here
\frame{
  \titlepage

  \vspace{-2em}

  \centerline{\href{https://goo.gl/i3CxL9}{\tt \scriptsize
                                           \underline{slides}: goo.gl/i3CxL9}}

  \vfill \hfill \includegraphics[height=6mm]{Figs/cc-zero.png} \vspace*{-0.5cm}

  \note{This slide deck is for a reading group presentation on the manuscript
    ``Fair Inference on Outcomes'' (Rabi \& Shpitser, 2017), for the seminar on
    ``Fairness in Machine Learning'', organized in Fall 2017 by Moritz Hardt, at
    the University of California, Berkeley.

    Source: {\tt https://github.com/nhejazi/talk\_fair-outcomes} \\
    Slides: {\tt https://goo.gl/i3CxL9} \\
    With notes: {\tt https://goo.gl/8RWEy5)}
}
}
}



\begin{frame}[c]{Preview: Summary}
\only<1>{\addtocounter{framenumber}{-1}}

\begin{center}
\begin{itemize}
  \itemsep12pt
  \item Penalization of specific pathways in DAGs provides an avenue to encode
    human intuition on fairness in learning algorithms.
  \item Mediation analysis in causal inference gives a general framework through
    which fairness intuitions may be expressed.
  \item point here
  \item anything else?
  \item point here
\end{itemize}
\end{center}

\note{We'll go over this summary again at the end of the talk. Hopefully, it
  will all make more sense then.
}

\end{frame}



\begin{frame}[fragile,c]{}

\begin{center}
\begin{minipage}[c]{9.3cm}
\begin{semiverbatim}
\lstset{basicstyle=\normalsize}
\begin{lstlisting}[linewidth=9.3cm]
  It's always good to start with a
  motivating example. Excerpts from
  famous studies/papers or personal
  communications usually do nicely.

  It's also good practice to keep
  things like this rather short.

  --Nima
\end{lstlisting}
\end{semiverbatim}
\end{minipage}
\end{center}

\note{Obviously, it's important to explain the motivating example here.
}
\end{frame}



\begin{frame}[c]{Preliminaries: Notation}

\begin{center}
\begin{itemize}
  \item ...
  \item ...
\end{itemize}
\end{center}

\note{...}

\end{frame}



\begin{frame}[c]{Preliminaries: Mediation Analysis}

\begin{center}
\begin{itemize}
  \item ...
  \item ...
\end{itemize}
\end{center}

\note{...}

\end{frame}



\begin{frame}[c]{The Average Causal Effect (ACE)}

\begin{center}
\begin{itemize}
  \item ...
  \item ...
\end{itemize}
\end{center}

\note{...}

\end{frame}



\begin{frame}[c]{The Natural \textit{Direct} Effect (NDE)}

\begin{center}
\begin{itemize}
  \item ...
  \item ...
\end{itemize}
\end{center}

\note{...}

\end{frame}



\begin{frame}[c]{The Natural \textit{Indirect} Effect (NIE)}

\begin{center}
\begin{itemize}
  \item ...
  \item ...
\end{itemize}
\end{center}

\note{...}

\end{frame}



\begin{frame}[c]{Introducing the Method or Approach, Part I}

\begin{center}
\begin{itemize}
  \item ...
  \item ...
\end{itemize}
\end{center}

\note{...}

\end{frame}



\begin{frame}[c]{Introducing the Method or Approach, Part II}

\begin{center}
\begin{itemize}
  \item ...
  \item ...
\end{itemize}
\end{center}

\note{...}

\end{frame}



\begin{frame}[c]{Introducing the Method or Approach, Part III}

\begin{center}
\begin{itemize}
  \item ...
  \item ...
\end{itemize}
\end{center}

\note{...}

\end{frame}



\begin{frame}[c]{Discussion of Results, Part I}

\begin{center}
\begin{itemize}
  \item ...
  \item ...
\end{itemize}
\end{center}

\note{...}

\end{frame}



\begin{frame}[c]{Discussion of Results, Part II}

\begin{center}
\begin{itemize}
  \item ...
  \item ...
\end{itemize}
\end{center}

\note{...}

\end{frame}



\begin{frame}[c]{Review: Summary}

\begin{center}
\begin{itemize}
  \itemsep12pt
  \item Look, we proved this above.
  \item More stuff we proved.
  \item Yet another point here.
  \item Final point goes here.
\end{itemize}
\end{center}

\note{It's always good to include a summary.}

\end{frame}



\begin{frame}[c,allowframebreaks]{References}
\bibliographystyle{apalike}
\nocite{*}
\bibliography{references}

\end{frame}



\begin{frame}[c]{}

\Large
Slides: \href{https://goo.gl/i3CxL9}{goo.gl/i3CxL9} \quad
\includegraphics[height=5mm]{Figs/cc-zero.png}

\vspace{5mm}

Notes: \href{https://goo.gl/8RWEy5}{goo.gl/8RWEy5}

\vspace{5mm}

\href{https://www.stat.berkeley.edu/~nhejazi}{\tt stat.berkeley.edu/\textasciitilde{}nhejazi}

\vspace{5mm}

\href{http://nimahejazi.org}{\tt nimahejazi.org}

\vspace{5mm}

\href{https://twitter.com/nshejazi}{\tt twitter/@nshejazi}

\vspace{5mm}

\href{https://github.com/nhejazi}{\tt github/nhejazi}

\note{Here's where you can find me, as well as the slides for this talk.}

\end{frame}



\end{document}

